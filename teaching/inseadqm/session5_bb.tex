\documentclass[xcolor=dvipsnames, 9pt]{beamer} % dvipsnames gives more built-in colors

% theme and color settings
\usetheme{Madrid}
\useoutertheme[hideothersubsections, left, height=0pt]{sidebar}
\useinnertheme{circles}

\definecolor{insead_dark}{RGB}{0,110,91} 
\definecolor{insead_light}{RGB}{160,206,103}
\definecolor{fade}{HTML}{C1C7D0}
\definecolor{light_red}{HTML}{DCBCBC}
\definecolor{mid_red}{HTML}{B97C7C}
\definecolor{dark_red}{HTML}{8F2727}

\usecolortheme[named=insead_dark]{structure}
\usefonttheme[onlymath]{serif}

% some packages 
\usepackage{xcolor}
\usepackage{graphicx} 
\usepackage{multicol}
\usepackage{tikz}
\usepackage{setspace}
\usepackage{tcolorbox}
\usepackage{colortbl}

%frontmatter
\title[Session 5 (Uncertainty)]{\normalsize{MBA Business Foundations, \\Quantitative Methods: \\Session Five}}
\author[Boris Babic, INSEAD]{Boris Babic, \\Assistant Professor of Decision Sciences}
\institute[]{}
\date{}

%\beamerdefaultoverlayspecification{<+->}

\begin{document}

\setbeamertemplate{sidebar left}{}
\begin{frame}
\titlepage
\vspace{-5em}
\begin{center}
\includegraphics[scale=0.4]{/Users/borisbabic/_INSEAD_teaching/_business_foundations/_syllabus/insead_logo.png}
\end{center}
\end{frame}

\begin{frame}
\frametitle{Today}
\begin{scriptsize}
\begin{tabular}{ll}
\textcolor{fade}{Basics} & \begin{tabular}[c]{@{}l@{}}\textcolor{fade}{Functions}\\ \textcolor{fade}{Linear}\\ \textcolor{fade}{Inverse}\\ \textcolor{fade}{Two equations}\\ \textcolor{fade}{Quadratic}\end{tabular} \vspace{0.25em} \\ 

\textcolor{fade}{Exponents}   & \begin{tabular}[c]{@{}l@{} }\arrayrulecolor{fade}\hline \\ \textcolor{fade}{ Exponents}\\ \textcolor{fade}{ Application: interest rates}\\ \textcolor{fade}{Exponential functions}\\ \textcolor{fade}{Logarithmic functions}\end{tabular} \vspace{0.25em} \\

\textcolor{fade}{Logarithms} & \begin{tabular}[c]{@{}l@{}} \\ \arrayrulecolor{fade}\hline \\ \textcolor{fade}{Logarithmic functions}\\ \textcolor{fade}{Logarithmic and exponential equations} \\ \textcolor{fade}{Case: pricing}\\ \textcolor{fade}{Derivatives} \end{tabular}   \vspace{0.25em}  \\

\textcolor{fade}{Derivatives}  & \begin{tabular}[c]{@{}l@{}} \arrayrulecolor{fade}\hline \\ \textcolor{fade}{Optimal decisions}\\ \textcolor{fade}{Case: production}\\\textcolor{fade}{ Statistics}\end{tabular}        \vspace{0.25em} \\
\textcolor{black}{Uncertainty}    & \begin{tabular}[c]{@{}l@{}} \arrayrulecolor{fade}\hline \\\textcolor{black}{ Probability \& statistics}\\ \textcolor{black}{Normal distribution }\end{tabular}     \vspace{0.25em} \\       
\end{tabular}
\end{scriptsize}
\end{frame}

\setbeamertemplate{sidebar left}[sidebar theme]
\section{Review from last class}

\begin{frame}
\frametitle{\textsf{R} vs. Excel}
\begin{itemize}
\item \textcolor{red}{Steeper learning curve} (whether it's worth it, depends on your objectives!)
\item Fast
\item Free
\item Flexible
\item Reproducible 
\item Extremely well supported (stackexchange, rbloggers, etc.)
\item Big data friendly (eg. no bound on rows/columns)
\item Visualization and graphics! (with ggplot2)
\item Data science (with the Tidyverse)
\item Machine learning (with Caret, randomForest, nnet, e1071)
\item Bayesian modeling/Markov Chain Monte Carlo (with Bugs, Jags, and Stan)
\item[] ... you learn as you go/as needed, shouldn't try to ``learn \textsf{R}"!
\end{itemize}
\end{frame}

\begin{frame}
\frametitle{Summary so far}
\begin{itemize}
\item Into to functions (ordinary, inverse, quadratic...)
\item Exponents and logs 
\item Derivatives 
\item Optimization 
\item Statistics and probability (last class: measures of location and dispersion)
\end{itemize}

\end{frame}



\section{Probability}

\begin{frame}
\frametitle{Probability}
\begin{center}
\includegraphics[scale=0.2]{tyche.png}
\end{center}
\begin{small}
\begin{itemize}
	\setlength\itemsep{1em}
	\item Tyche: Goddess of chance (daughter of Zeus). 
	\item The ancient Greeks believed that when no other cause can be attributed to random events such as floods, droughts, frosts, then Tyche is responsible. 
	\item Probability is the study of such random events or, more generally, of randomness. 
\end{itemize}
\end{small}
\end{frame}

\begin{frame}
\frametitle{Probability}
\vspace{-1em}
\begin{center}
\includegraphics[scale=0.3]{random_events.png}
\end{center}
\begin{small}
\vspace{-2em}
\begin{itemize}
	\setlength\itemsep{1em}
	\item But what is probability? 
	\item For our purposes, the probability of an event can be interpreted as the long-run frequency of the occurrence of that event. 
	\item \textcolor{red}{Always???} 
	\item What is the probability of a nuclear war in the next year? Of US dollar collapse? 
	\item When Galileo first observed Saturn through a telescope, he saw something like this. 
	\item[] \begin{center}
	\vspace{-0.5cm}
\includegraphics[scale=0.5]{galileo.png}
\end{center}
	\item Are those rings around the planet? Handles? Or is it three planets next to each other? Can you assign a probability? 
\end{itemize}
\end{small}
\end{frame}

\begin{frame}
\frametitle{Probability vs. statistics}
\begin{itemize}
	\setlength\itemsep{1em}
	\item Revisit: What is the difference between probability and statistics?
	\item A probability question: A \textcolor{dark_red}{fair} die will be tossed twice. What is the probability that it lands on six both times? 
	\item A (descriptive) statistics question: A die was tossed twice and it landed on a five and a six. What is the mean die value? 
	\item An (inferential) statistics question: A die was tossed twice and it landed on a five and a six. How confident are you that the die is \textcolor{dark_red}{fair}? 
	\item To answer this question, can you use descriptive statistics? Can you use probability? How? Is there a right answer? What is it? 
\end{itemize}
\end{frame}

\begin{frame}
\frametitle{Random variables}

\begin{itemize}
	\setlength\itemsep{1em}
	\item In probability we know which values our variables $X$ can take, and we know how probable those values are. 
	\item Ex: if the variable represents the outcome of the toss of a fair die (i.e., which face landed up), what are the values? How likely are they? 
	\item Variables are 
		\begin{itemize}
			\item Discrete, corresponding to natural or counting numbers. 
			\item Continuous, corresponding to real numbers. 
		\end{itemize}
	\item Classify the following: 
		\begin{itemize}
			\item Height or weight
			\item Number of monthly lottery winners in California
			\item Temperature tomorrow 
			\item A parent's number of children 
			\item The amount of money in your bank account
		\end{itemize}
\end{itemize}
\end{frame}

\begin{frame}
\frametitle{Knowing your sample space}
\begin{itemize}
\item A certain couple has two children. At least one of them is a boy. What is the probability that both children are boys?
\item Possibilities: BB, BG, GB, GG 
\item What can we rule out? GG 
\item What remains: BB, BG, GB
\item Probability that both children are boys is 1/3. 
\end{itemize}
\end{frame}

\begin{frame}
\frametitle{Probability distributions}
\begin{small}
\begin{itemize}
	\itemsep\setlength{1em}
\item A probability distribution is a function $f$ of the random variable $X$: $f(X)$
\item This function can only take values between $0$ and $1$.
\item Also, this function is additive: for two independent events, the probability of their sum is the sum of their probabilities. 
\item Ex: Pr (die lands on 4) + Pr (die lands on 6) = Pr (die lands on 4  or die lands on 6).
\item In the discrete case, it tells us how probable it is that the random variable $X$ will take a specific value $x$. We often denote the function $f$ with $\textrm{Pr.}$
\item Ex: For a fair die, $f(3) = \textrm{Pr}(X = 3) = 1/6$.
\item In the continuous case, it tells us how likely it is that our variable will be contained within an interval $[a, b]$. 
\item Ex: For a person's weight, $\textrm{Pr}(a < x < b) = k$.
\item But what about $\textrm{Pr}(x)$ in the continuous case (???).
\end{itemize}
\end{small}
\end{frame}

\begin{frame}[fragile]
\frametitle{Variations on exercise problem 4}
\begin{itemize}
\itemsep\setlength{1em}
	\item A fair die is rolled once. 
	\item What is the probability that it lands on a number greater than 4?
	\item[] $\textrm{Pr}(X > 4) = \textrm{Pr}(X = 5) + \textrm{Pr}(X = 6) = 2/6 = 1/3$
	\item[] 
	\begin{verbatim}
	library(dice)
	getEventProb(nrolls = 1,ndicePerRoll = 1,
		nsidesPerDie = 6,eventList = list(5:6))
	[1] 0.3333333
		\end{verbatim}
	\item A fair die is rolled 5 times. What is the probability of seeing exactly the pattern 6, 5, 4, 3, followed by a 2 or a 1? 
	\item $\bigg( \frac{1}{6} \bigg)^5 + \bigg( \frac{1}{6} \bigg)^5 = 0.0002$
	\item[]
	\begin{verbatim}
	library(dice)
	getEventProb(nrolls = 5,ndicePerRoll = 1,nsidesPerDie = 6,
	eventList = list(6, 5, 4, 3, 1:2),orderMatters = TRUE)
	[1] 0.0002572016
	\end{verbatim}
\end{itemize}
\end{frame}

\begin{frame}
\frametitle{Doing things manually vs. automating them: geek vs. non-geek}
\begin{center}
\includegraphics[scale=0.35]{r_geeks.png}
\end{center}
\end{frame}

\begin{frame}
\frametitle{Birthdays}
\begin{itemize}
	\itemsep\setlength{1em}
\item We have 30 students in this class. What is the probability that at least one pair of students share the same birthday?
\item[]
\item Algebraic solution:
\item[] $\frac{30 \cdot 29}{2} = 435$ pairs of students
\item[] $\frac{364}{365}$ probability that a single pair does not share a birthday
\item[] $\bigg( \frac{364}{365} \bigg)^{435} = 0.30$ probability that no pair shares a birthday
\item[] $1 - 0.30 = 0.70$ probability that at least one pair shares a birthday
\end{itemize}
\end{frame}

\begin{frame}[fragile]
\frametitle{Birthdays: \textsf{R} solution}
\begin{center}
\begin{scriptsize}
\begin{verbatim}
k = 30  
p <- numeric(k)  # create numeric vector to store probabilities
for (i in 1:k)      {
            q <- 1 - (0:(i - 1))/365  # 1 - prob(no matches)
            p[i] <- 1 - prod(q)  }
prob <- p[k]
print(prob)
#BONUS: 

plot(p, main="Probability at least 2 people have same Birthday", 
xlab ="Number of People", ylab = "Probability", col="blue")
[1] 0.7063162
\end{verbatim}
\end{scriptsize}
\includegraphics[scale=0.27]{birthday.png}
\end{center}
\end{frame}

\section{Normal distribution}

\begin{frame}
\frametitle{Normal distribution}
\begin{itemize}
\item Most important probability distribution you will encounter (due, in part, to the central limit theorem).
\item This distribution belongs to the exponential family of distributions, and it has two parameters, its average $\mu$ and standard deviation $\sigma$. 
\item Represented by the famous ``bell curve'': symmetric around its mean
\item[] \begin{center}
\includegraphics[scale=0.45]{normal_curve.pdf}
\end{center}
\item Given by $$ \textcolor{dark_red}{ f(x| \mu, \sigma^2) = (2 \pi \sigma^2)^{-1/2} e^{ \frac{-(x - \mu)^2}{2 \sigma^2} } }$$
\item Bonus: Why is $\pi$ in there? Because $\int_{- \infty}^{\infty} e^{-x^2} dx = \sqrt{\pi}$!
\end{itemize}
\end{frame}

\begin{frame}
\frametitle{Normal distribution}
\begin{center}
\includegraphics[scale=0.3]{normal_curve2.png}
\begin{itemize}
\item[] The probability that our random variable is between $a$ and $b$ is given by the area under the curve between those two points: $$ \textrm{Pr}(a < x < b) = (2 \pi \sigma^2)^{-1/2} \int_a^b e^{ \frac{-(x - \mu)^2}{2 \sigma^2} } dx$$
\end{itemize}
\end{center}
\end{frame}

\begin{frame}
\frametitle{Normal distribution}
\begin{center}
\includegraphics[scale=0.3]{histogram_3.png}
\begin{itemize}
\item Variables that have a normal distribution are ubiquitous in real life, provided we have enough data.
\item Ex: Age of INSEAD students, height of INSEAD students.
\end{itemize}
\end{center}
\end{frame}

\begin{frame}
\frametitle{Normal distribution}
\begin{center}
\includegraphics[scale=0.35]{normal_means.pdf}
\begin{scriptsize}
\begin{itemize}
\item As the mean changes, the location of the bell shifts
\item[] To the left (for smaller means)
\item[] To the right (for larger means)
\end{itemize}
\end{scriptsize}
\end{center}
\begin{center}
\includegraphics[scale=0.35]{normal_variances.pdf}
\begin{scriptsize}
\begin{itemize}
\item As the standard deviation changes, the bell becomes
\item[] taller and thinner (for smaller standard deviations)
\item[] shorter and thicker (for larger standard deviations)
\end{itemize}
\end{scriptsize}
\end{center}
\end{frame}

\begin{frame}
\frametitle{Normal distribution}
\begin{itemize}
	\itemsep\setlength{1em}
\item So how to compute these areas under the curve (=probabilities)? 
\item The integral does not have a closed form. 
\item Rescale to a standard normal distribution and then use a table.
\item Or, use computational approach, for example in \textsf{R}! 
\end{itemize}
\begin{center}
\includegraphics[scale=0.5]{bell_curve.png}
\end{center}
\end{frame}


\begin{frame}
\frametitle{Transformations}
\begin{itemize}
	\itemsep\setlength{1em}
	\item A standard normal distribution is a normal distribution with mean $0$ and variance $1$.
	\item The distribution function of a standard normal is given by $$ (2 \pi)^{-1/2} e^{ \frac{-x^2}{2} }$$
	\item To denote that a random variable $X$ has a normal distribution we will use $X \sim \textrm{N}(\mu, \sigma^2)$.
	\item \textcolor{dark_red}{if} $X$ follows a normal distribution with mean $\mu$ and standard deviation $\sigma$, $X \sim \textrm{N}(\mu, \sigma^2)$, \textcolor{dark_red}{then} $Z = \frac{x - \mu}{\sigma}$ follows a standard normal distribution, $Z \sim \textrm{N}(0, 1)$.
\end{itemize}
\end{frame}

\begin{frame}
\frametitle{Transformations}
\begin{center}
\includegraphics[scale=0.5]{z1.png}
\end{center}
\vspace{-2em}
\begin{itemize}
\item Capital $F$ denotes a cumulative distribution: $$F(z) = Pr (Z \leq z) = \int_{- \infty}^z f(z) dz$$
\item So $\textrm{Pr}(a < x < b) = F(b) - F(a)$
\end{itemize}
\end{frame}

\begin{frame}
\frametitle{Transformations}
\begin{center}
\includegraphics[scale=0.5]{z2.png}
\end{center}
\end{frame}

\begin{frame}
\frametitle{Transformations}
\begin{center}
\includegraphics[scale=0.5]{z3.png}
\end{center}
\end{frame}

\begin{frame}
\frametitle{Computing normal probabilities}
\begin{itemize}
\item Ex: You want $\textrm{Pr}(X > k)$ where $X \sim \textrm{N}(\mu, \sigma)$. 
\item \textcolor{dark_red}{Step 1}: Transform $X \rightarrow Z$
\item[]
\begin{align*}
\textrm{Pr}(X > k)
&= \textrm{Pr} \bigg( \frac{X - \mu}{\sigma} > \frac{k - \mu}{\sigma} \bigg) \\
&= \textrm{Pr} \bigg( Z > \frac{k - \mu}{\sigma} \bigg)
\end{align*}
\item \textcolor{dark_red}{Step 2}: Look up the probability of $F(\frac{k - \mu}{\sigma})$ in \textsf{R} or a standard normal table. 
\item \textcolor{dark_red}{Step 3}: Get the result: $$ \textrm{Pr}(X > k) = F( \frac{k - \mu}{\sigma}) $$
\end{itemize}
\end{frame}

\begin{frame}
\frametitle{Examples}
\begin{itemize}
\item Suppose that the test scores of a course exam at INSEAD are normally distributed with a mean of $72$ and a standard deviation of $15.2$. What is the probability that a randomly chosen student received above $84$?
\item[]
$$ \textrm{Pr}(X > 84) \rightarrow \textrm{Pr} \bigg( \frac{X - \mu}{\sigma} > \frac{84 - 72}{15.2} \bigg) \rightarrow \textrm{Pr}(Z > 0.789) $$
\item \textcolor{blue}{pnorm(0.789, lower.tail=FALSE)}
\item \textcolor{blue}{1-pnorm(0.789)}
\item OR, you can avoid the transformations altogether! \textcolor{blue}{pnorm(84, mean=72, sd=15.2, lower.tail=FALSE)}
\item Approximately $21\%$.
\item We use lower.tail=FALSE in order to get the area from $x$ to $\infty$. 
\end{itemize}
\end{frame}

\begin{frame}
\frametitle{The pnorm() function}
\begin{itemize}
\item The \textcolor{blue}{pnorm} function replaces the lookup table at the end of all statistics textbooks. 
\item \textcolor{blue}{pnorm} returns the integral from $-\infty$ to $k$ of the pdf of the normal distribution. That is, $F(k)$. 
\item If you do not set any further values, then $k$ is a $Z$-score by default. 
\item However, you can specify the mean and variance as \textcolor{blue}{pnorm(2, mean = 5, sd = 3)}
\end{itemize}
\end{frame}

\begin{frame}
\frametitle{Practice}
\begin{itemize}
\item The weekly salaries of the employees of a large corporation are assumed to be normally distributed with mean \$450 and standard deviation \$40.
\item What is the probability that a randomly chosen employee earns more than \$500 per week?
\end{itemize}
\end{frame}

\begin{frame}
\frametitle{Solution}
\begin{itemize}
\item Find the Z score and look it up:
\item[] $$ \textrm{Pr}(X > 500) = \textrm{Pr} \bigg( \frac{X - 450}{40} > \frac{500-450}{40} \bigg) = \textrm{Pr} \bigg( Z > \frac{5}{4} \bigg) $$
\vspace{1em}
\item \textcolor{blue}{pnorm(500, mean=450, sd=40, lower.tail=FALSE)}
\item approximately $10\%$.
\end{itemize}
\end{frame}

\begin{frame}
\frametitle{Tips}
\begin{itemize}
	\itemsep\setlength{1em}
\item Probabilities correspond to areas
\item Probabilities sum to 1: $ \textrm{Pr}(Z < k) = 1 - \textrm{Pr}(Z > k) $
\item Symmetry: $\textrm{Pr}(Z < -k) = \textrm{Pr}(Z > k)$
\item For intervals, use substraction: $\textrm{Pr}(a < Z < b) = \textrm{Pr}(Z > a) - \textrm{Pr}(Z > b)$
\end{itemize}
\end{frame}

\begin{frame}
\frametitle{Practice: but how much money will I make??? }
\begin{itemize}
\item The average post INSEAD MBA starting salary is $170k$ with a standard deviation of $30k$. 
\item You want to know the probability of earning between $150$ to $200k$ upon graduation.
\item So you want to calculate the probability that a randomly chosen post-MBA INSEAD student starts at $150-200k$. 
\item Find the answer algebraically, then confirm in \textsf{R}.
\end{itemize}
\end{frame}

\begin{frame}[fragile]
\frametitle{Solution}
Algebraic:
\begin{align*}
\textrm{Pr}(150 < x < 200)
&= \textrm{Pr}(x > 150) - \textrm{Pr}(x < 200) \\
&= \textrm{Pr} \bigg( \frac{x - 170}{30} > \frac{150 - 170}{30} \bigg) - \bigg( \frac{x - 170}{30} > \frac{200 - 170}{30} \bigg) \\
&= \textrm{Pr} \bigg( Z > \frac{150 - 170}{30} \bigg) - \bigg( Z > \frac{200 - 170}{30} \bigg) \\
&= \textrm{Pr} \bigg( Z > -\frac{2}{3} \bigg) - \bigg( Z > 1 \bigg) \\
&= .7454 - .1587 \\
&= .5867
\end{align*}
In \textsf{R}: 
\begin{verbatim}
pnorm(150, mean=170, sd=30, lower.tail=FALSE) - 
pnorm(200, mean=170, sd=30, lower.tail=FALSE)
\end{verbatim}
\end{frame}

\begin{frame}[fragile]
\frametitle{The rnorm() function}
\begin{verbatim}
x <- rnorm(10000, mean=100, sd=15)
hist(x, probability=TRUE)
xx <- seq(min(x), max(x), length=150)
lines(xx, dnorm(xx, mean=100, sd=15))
\end{verbatim}
\begin{center}
\includegraphics[scale=0.25]{rnorm.png}
\end{center}
This generates 10000 random numbers from a specified normal distribution (first line), plots their histogram (second line), and graphs the distribution function of the same normal distribution (third and fourth lines).
\end{frame}

\section{MDM}

\begin{frame}
\frametitle{Management decision making}

\begin{itemize}
\item I will teach the MBA elective Management Decision Making (or MDM) in P4 (March/April)!
\item You should all take it! Tell your friends too!
\item Based on prototype I designed at Caltech called ``Statistics, Ethics and Law''
\end{itemize}
\end{frame}

\begin{frame}
\frametitle{Management decision making}
\hspace{-0.5cm} \includegraphics[scale=0.21]{advertisement.png} 
\end{frame}

\begin{frame}
\frametitle{Management decision making}
\begin{itemize}
\item[] Some topics we will look at:
\item \textcolor{dark_red}{Human bias}
\begin{itemize}
\item Risk and rational choice
\item Statistical reasoning (p-hacking, small samples, overfitting)
\item Implicit bias, anchoring, loss aversion, ambiguity
\item Moral reasoning
\end{itemize}
\item \textcolor{dark_red}{Machine bias}
\begin{itemize}
\item Algorithmic justice  (ex: facial recognition systems)
\item Discrimination in lending, credit scoring, policing
\item Bias in advertising
\item Transparency in algorithmic decision-making (cf: adversarial attacks)
\item Privacy in data collection (for example in health care, on social media)
\item Ethics of self-driving cars and medical technology
\end{itemize}
\end{itemize}
\end{frame}

%pnorm(84, mean=72, sd=15.2, lower.tail=FALSE)
%endnotes
\begin{frame}
\frametitle{Today}
\begin{scriptsize}
\begin{tabular}{ll}
\textcolor{fade}{Basics} & \begin{tabular}[c]{@{}l@{}}\textcolor{fade}{Functions}\\ \textcolor{fade}{Linear}\\ \textcolor{fade}{Inverse}\\ \textcolor{fade}{Two equations}\\ \textcolor{fade}{Quadratic}\end{tabular} \vspace{0.25em} \\ 

\textcolor{fade}{Exponents}   & \begin{tabular}[c]{@{}l@{} }\arrayrulecolor{fade}\hline \\ \textcolor{fade}{ Exponents}\\ \textcolor{fade}{ Application: interest rates}\\ \textcolor{fade}{Exponential functions}\\ \textcolor{fade}{Logarithmic functions}\end{tabular} \vspace{0.25em} \\

\textcolor{fade}{Logarithms} & \begin{tabular}[c]{@{}l@{}} \\ \arrayrulecolor{fade}\hline \\ \textcolor{fade}{Logarithmic functions}\\ \textcolor{fade}{Logarithmic and exponential equations} \\ \textcolor{fade}{Case: pricing}\\ \textcolor{fade}{Derivatives} \end{tabular}   \vspace{0.25em}  \\

\textcolor{fade}{Derivatives}  & \begin{tabular}[c]{@{}l@{}} \arrayrulecolor{fade}\hline \\ \textcolor{fade}{Optimal decisions}\\ \textcolor{fade}{Case: production}\\\textcolor{fade}{ Statistics}\end{tabular}        \vspace{0.25em} \\
\textcolor{black}{Uncertainty}    & \begin{tabular}[c]{@{}l@{}} \arrayrulecolor{fade}\hline \\\textcolor{black}{ Probability \& statistics}\\ \textcolor{black}{Normal distribution }\end{tabular}     \vspace{0.25em} \\       
\end{tabular}
\end{scriptsize}
\end{frame}

    {\setbeamercolor{background canvas}{bg=insead_dark}
    \begin{frame}[plain]
    \begin{tikzpicture}[remember picture,overlay]
    \node[at=(current page.center)] {
    \includegraphics[keepaspectratio,
    width=\paperwidth,
    height=\paperheight]{/Users/borisbabic/_INSEAD_teaching/_business_foundations/_syllabus/end_slide.png}};
    \end{tikzpicture}
    \end{frame}}
\end{document}