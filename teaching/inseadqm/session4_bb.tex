\documentclass[xcolor=dvipsnames]{beamer} % dvipsnames gives more built-in colors

% theme and color settings
\usetheme{Madrid}
\useoutertheme[hideothersubsections, left, height=0pt]{sidebar}
\useinnertheme{circles}

\definecolor{insead_dark}{RGB}{0,110,91} 
\definecolor{insead_light}{RGB}{160,206,103}
\definecolor{fade}{HTML}{C1C7D0}
\definecolor{light_red}{HTML}{DCBCBC}
\definecolor{mid_red}{HTML}{B97C7C}
\definecolor{dark_red}{HTML}{8F2727}

\usecolortheme[named=insead_dark]{structure}
\usefonttheme[onlymath]{serif}

% some packages 
\usepackage{xcolor}
\usepackage{graphicx} 
\usepackage{multicol}
\usepackage{tikz}
\usepackage{setspace}
\usepackage{tcolorbox}
\usepackage{colortbl}
\usepackage{verbatim}
\usepackage{listings}

%frontmatter
\title[Session 4 (Derivatives)]{\normalsize{MBA Business Foundations, \\Quantitative Methods: \\Session Four}}
\author[Boris Babic, INSEAD]{Boris Babic, \\Assistant Professor of Decision Sciences}
\institute[]{}
\date{}

%\beamerdefaultoverlayspecification{<+->}

\begin{document}

\setbeamertemplate{sidebar left}{}
\begin{frame}
\titlepage
\vspace{-5em}
\begin{center}
\includegraphics[scale=0.4]{/Users/borisbabic/_INSEAD_teaching/_business_foundations/_syllabus/insead_logo.png}
\end{center}
\end{frame}

\begin{frame}
\frametitle{Today}
\begin{scriptsize}
\begin{tabular}{ll}
\textcolor{fade}{Basics} & \begin{tabular}[c]{@{}l@{}}\textcolor{fade}{Functions}\\ \textcolor{fade}{Linear}\\ \textcolor{fade}{Inverse}\\ \textcolor{fade}{Two equations}\\ \textcolor{fade}{Quadratic}\end{tabular} \vspace{0.25em} \\ 

\textcolor{fade}{Exponents}   & \begin{tabular}[c]{@{}l@{} }\arrayrulecolor{fade}\hline \\ \textcolor{fade}{ Exponents}\\ \textcolor{fade}{ Application: interest rates}\\ \textcolor{fade}{Exponential functions}\\ \textcolor{fade}{Logarithmic functions}\end{tabular} \vspace{0.25em} \\

\textcolor{fade}{Logarithms} & \begin{tabular}[c]{@{}l@{}} \\ \arrayrulecolor{fade}\hline \\ \textcolor{fade}{Logarithmic functions}\\ \textcolor{fade}{Logarithmic and exponential equations} \\ \textcolor{fade}{Case: pricing}\\ \textcolor{fade}{Derivatives} \end{tabular}   \vspace{0.25em}  \\

\textcolor{black}{Derivatives}  & \begin{tabular}[c]{@{}l@{}} \arrayrulecolor{fade}\hline \\ \textcolor{black}{Optimal decisions}\\ \textcolor{black}{Case: production}\\\textcolor{black}{ Statistics}\end{tabular}        \vspace{0.25em} \\

\textcolor{fade}{Uncertainty}    & \begin{tabular}[c]{@{}l@{}} \arrayrulecolor{fade}\hline \\\textcolor{fade}{ Probability \& statistics}\\ \textcolor{fade}{Normal distribution }\end{tabular}     \vspace{0.25em} \\       
\end{tabular}
\end{scriptsize}
\end{frame}

\setbeamertemplate{sidebar left}[sidebar theme]
\section{Optimal decisions}

\begin{frame}
\frametitle{Analyzing functions with derivatives}

\begin{itemize}
    \setlength\itemsep{1em}
\item We are given some function $f(x)$ and want to know something about its behavior at $x_1$.
\item Find $f'(x)$.
\item Find $f'(x_1)$.
\item If $f'(x_1) > 0$ the function is increasing ``at that point''.
\item $f'(x_1) < 0$ the function is decreasing ``at that point''.
\item if for all $x, f'(x) > 0$ the function is increasing.
\item if for all $x, f'(x) < 0$ the function is decreasing.
\item $f'(x_1) = 0$ the function is at a maximum or minimum (most likely). 
\end{itemize}
\end{frame}

\begin{frame}
\frametitle{Example}
Consider the function $f(x) = 2x^3 + x^2 - 4x - 3$

\begin{center}
\includegraphics[scale=0.5]{def_example.pdf}
\end{center}

\begin{itemize}
\item Find $f'(x)$. 
\item Evaluate $x$ at $(-2, -1, 0, 1, 1.5)$.
\end{itemize}
\end{frame}

\begin{frame}
\frametitle{Solution}
\begin{itemize}
\item $f'(x) = 6x^2 + 2x - 4$
\item $f'(-2) = 16$
\item $f'(-1) =  0$
\item $f'(0) =  -4$
\item $f'(1) =  4$
\item $f'(1.5) = 12.5$
\end{itemize}
\end{frame}

\begin{frame}
\frametitle{Second derivatives}
\begin{itemize}
\item Consider a function which takes time as its input and gives us a car's distance traveled as its output. 
\item The first derivative of such a function corresponds to the car's velocity. 
\item The second derivative would be the derivative of the derivative. 
\item This corresponds to the car's acceleration. 
\item It measures how the rate of change is itself changing. 
\item Graphically, this corresponds to a function's curvature -- degree of concavity/convexity.  
\item Formally, there is nothing new -- to take the second derivative, treat the derivative function as your original function, and apply the rules from last class!
\end{itemize}
\end{frame}

\begin{frame}
\frametitle{Concavity/convexity}
\begin{small}
\begin{itemize}
\item A function is called convex on an interval $[a, b]$ if the line segment between any two points on the graph of the function over that interval lies above or on the graph. Ex: $x^2$.
\item If such line segment is below the graph of the function, it is concave. 
\item Important wherever ``marginal" values are relevant -- utility, revenue, economies of scale, etc. 
\item We denote the second derivative of $f(x)$ as $f''(x)$ or $$\frac{d}{dx} \frac{d}{dx} f(x) = \frac{d^2}{dx^2} f(x)$$
\item \textcolor{dark_red}{If $f''(x) > 0$ for all $x$ in the interval $[a, b]$, then $f(x)$ is convex on $[a, b]$. Ex: $x^2$.}
\item \textcolor{dark_red}{If $f''(x) < 0$ for all $x$ in the interval $[a, b]$, then $f(x)$ is concave on $[a, b]$. Ex: $x^{1/2}$.}
\end{itemize}
\end{small}
\end{frame}

\begin{frame}
\frametitle{Maximum/minimum of a function}
\begin{scriptsize}
To find potential max-min points of a function $f(x)$:
\begin{itemize}
    \setlength\itemsep{1em}
    \item Compute the (first) derivative $f'(x)$
    \item Solve the equation $f'(x) = 0$. The points $x^*$ obtained are possible candidates for maxima/minima. %but need to be checked. 
    \item[] \textcolor{dark_red}{$\longrightarrow$ First Order Condition (FOC)}
    \item Compute the second derivative $f''(x)$ %(derivative of derivative)
    \item If $f''(x^*) > 0$ then $x^*$ is a (local) minimum
    \item[] If this is true for all $x$, then global minimum
    \item If $f''(x^*) < 0$ then $x^*$ is a (local) maximum
    \item[] If this is true for all $x$ then global maximum
    \item[] \textcolor{dark_red}{$\longrightarrow$ Second Order Condition (SOC)}
\end{itemize}
\end{scriptsize}
\end{frame}

\begin{frame}
\frametitle{Example}
Identify the local maximum/minimum of the function: $$ f(x) = 2x^3 + x^2 - 4x -3 $$
\begin{center}
\includegraphics[scale=0.5]{def_example.pdf}
\end{center}
\end{frame}

\begin{frame}
\frametitle{Solution}
\begin{itemize}
    \itemsep\setlength{1em}
\item $f(x) = 2x^3 + x^2 - 4x - 3$
\item $f'(x) = 6x^2 + 2x - 4$
\item $f''(x) = 12x + 2$
\item FOC: $6x^2 + 2x - 4 = 0 \rightarrow x^* = -1, ~\textrm{and}~ x^* = 2/3$.
\item SOC: 
\item[] $12(-1) + 2 < 0$ (-1 is a maximum)
\item[] $12(2/3) + 2 > 0$, (2/3 is a minimum). 
\end{itemize}
\end{frame}

%\begin{frame}
%\frametitle{Global minima/maxima}
%\begin{center}
%\hspace*{-0.6cm} \includegraphics[scale=0.5]{global_extrema.png}
%\end{center}
%\begin{scriptsize}
%\begin{multicols}{2}
%\begin{itemize}
%\itemsep\setlength{1em}
%\item $f$ has one global max. at $x_0$ if:
%\item[] $f'(x_0) = 0$ FOC  
%\item[] $f''(x) > 0$ $\forall \, x$ ($f$ is convex).
%\item $f$ has one global min. at $x_0$ if: 
%\item[] $f'(x_0) = 0$ (FOC) 
%\item[] $f''(x) < 0$ $\forall \, x$ ($f$ is concave). 
%\end{itemize}
%\end{multicols}
%\end{scriptsize}
%\end{frame}

\begin{frame}
\frametitle{Application to quadratic function}
\begin{center}
\includegraphics[scale=0.5]{quad_example.pdf}
\end{center}
\vspace{-1em}
\begin{scriptsize}
\begin{itemize}
    \setlength\itemsep{1em}

\item $f(x) = ax^2 + bx + c$
\item $f'(x) = 2ax + b \rightarrow ~\textrm{FOC:}~2ax + b = 0 \rightarrow x_0 = \frac{-b}{2a}$
\item $f''(x) = 2a$
\item if $a > 0$, $x_0 = \frac{-b}{2a}$ is global minimum (b/c $f''(x) = 2a >0$)
\item if $a < 0$, $x_0 = \frac{-b}{2a}$ is global maximum (b/c $f''(x) = 2a <0$)
\end{itemize}
\end{scriptsize}
\end{frame}

\begin{frame}
\frametitle{Application of derivatives in economics}
\begin{itemize}
    \itemsep\setlength{1em}
\item Consider profit ($p$) as a function of advertising cost ($c$). 
\item[] \textcolor{dark_red}{$p = f(c) = -c^2 + 3c -2 $}
\item[] At what level of advertising will the profit be maximized? 
\item Consider a demand function, with price ($p$) and quantity demanded ($q$).
\item[] \textcolor{dark_red}{$p = f(q) = 120 - 4q$}
\item[] Write revenue as a function of quantity demanded. 
\item[] To maximize revenue, how many units should we sell? 
\item[] Which price should we set?
\end{itemize}
\end{frame}

\begin{frame}
\frametitle{Solutions}
\begin{itemize}
\item Problem 1: $\arg \max f(c) = 3/2$ (you may often see this notation).
\item Problem 2:
\begin{itemize}
\item[a.] Revenue = price $\times$ quantity $= -4q^2 + 120q$ 
\item[b.] $-4q^2 + 120q = 0 \rightarrow q^* = 15$ units 
\item[c.] $f(q^*) = 120 - 4 \times 15 = 60 \rightarrow p^* = 60$
\end{itemize}
\end{itemize}
\end{frame}

\begin{frame}
\frametitle{Marginal profit}
\begin{scriptsize}
\begin{center}
\includegraphics[scale=0.25]{disc_mp.png}
\end{center}
\begin{itemize}
     \itemsep\setlength{1em}
\item Discrete case, where $\pi(q)$ is profit given $q$ units and $m \pi(q)$ marginal profit at $q$ : $ m\pi (q) = \pi (q + 1) - \pi (q)$
\item ``Profit earned by producing one unit above q''
\item Note this is \emph{also} the rate of change of $\pi(q)$ at $q$ $\rightarrow$ $\frac{\pi(q+1) - \pi(q)}{q + 1 - q}$
\item Marginal profit is given by the \textcolor{dark_red}{slope} of the profit function at $q$.
\end{itemize}
\end{scriptsize}
\end{frame}

\begin{frame}
\frametitle{Marginal profit}
\begin{scriptsize}
\begin{center}
\includegraphics[scale=0.25]{cont_mp.png}
\end{center}
\begin{itemize}
     \itemsep\setlength{1em}
\item Continuous case: $m \pi (q) = \pi'(q)$
\item[] ``Profit earned by producing \emph{a little more} than q''
\item[] [What is a little more?]
\end{itemize}
\end{scriptsize}
\end{frame}

\begin{frame}
\frametitle{Marginal functions}
Likewise, we can define for continuous revenue/costs: 
\begin{itemize}
    \itemsep\setlength{1em}
\item If $r = f(q)$ is the revenue function, then the marginal revenue (MR) is $r' = f'(q)$. 
\item If $c = f(q)$ is the cost function, then the marginal cost (MC) is $c' = f'(q)$.  
\end{itemize}
\end{frame}

\begin{frame}
\frametitle{Bonus example: marginal revenue}

\begin{itemize}
    \itemsep\setlength{1em}
\item Given the following demand function,
\item[] $p = -5q^2 + 30q + 7$
\item find the marginal revenue function, where price $=p$ and quantity $=q$.
\item What is the marginal revenue at $q = 2, 3, 5$?
\item For which $q$ do we maximize revenue? 
\end{itemize}

\end{frame}

\begin{frame}
\frametitle{Solution}
\begin{center}
\includegraphics[scale=0.5]{marg_profit.pdf}
\end{center}
\begin{itemize}
    \item $f'(q) = - 10x + 30$
    \item $f'(2) = 10, f'(3)=0, f'(5) = - 20$
    \item $\arg \max f(q) = 3$
\end{itemize}

\end{frame}

\begin{frame}
\frametitle{(Bonus) Overview of production planning}
\begin{itemize}
    \itemsep\setlength{1em}
\item Total profit = total revenue - total cost:
\item[] $TP(Q) = TR(Q) - TC(Q)$
\item Marginal profit = marginal revenue - marginal cost 
\item[] $MP(Q) = MR(Q) - MC(Q)$
\item $MR(Q) > MC(Q)$ $\rightarrow$ should produce more 
\item $MR(Q) < MC(Q)$ $\rightarrow$ should produce less
\item $MR(Q) = MC(Q)$ $\rightarrow$ optimal production, but check conditions for max
\end{itemize}
\end{frame}

\section{Case: production}

\begin{frame}
\frametitle{Case discussion}
Motor Cycle Helmets with Bluetooth (B) : Production
\end{frame}

\section{Statistics}

\begin{frame}
\frametitle{Statistics}
\begin{center}
\hspace*{-0.5cm} \includegraphics[scale=0.25]{stats_brands.png}
\end{center}
\end{frame}

\begin{frame}
\frametitle{The R statistical programming language}
\begin{scriptsize}
\begin{itemize}
    \itemsep\setlength{1em}
\item For the statistics section, we will periodically use the statistical programming language \textcolor{dark_red}{\textsf{R}} to perform some basic operations. 
\item \textcolor{dark_red}{\textsf{R}} is an open source language, written in C and Fortran. 
\item It was initially developed at Bell Labs, where it was known (creatively) as S. 
\item Today it is the most widely used language among statisticians. 
\item You can download a free distribution on the \textcolor{dark_red}{\textsf{R}} project web page (www.r-project.org/) together with RStudio (rstudio.com) which is the leading IDE for \textcolor{dark_red}{\textsf{R}} (basically a gui). 
\item In this class, however, we will use the online implementation rdrr: \href{https://rdrr.io/snippets/}{ \textcolor{red}{https://rdrr.io/snippets/} }
\item You do not need to download anything!
\item Does anyone know why it is called rdrr? See \href{https://www.youtube.com/watch?v=gSFd_2oJgak}{\textcolor{red}{here}}.
\end{itemize}
\end{scriptsize}
\end{frame}

\begin{frame}[fragile]
\frametitle{Measures of location}
\begin{scriptsize}
``If I was allowed one number to describe my dataset, what would it be?''
\begin{itemize}
\item Three notions: 
\item Mean: average value, $\mathrm{E}[X] = \frac{1}{n} [x_1 + x_2 + ... + x_n]$
\item[] \textcolor{dark_red}{\textsf{R}: mean()}
\item Median: the middle value when the dataset is ordered from smallest to largest
\item[] \textcolor{dark_red}{\textsf{R}: median()}
\item Mode: the value with highest frequency
\item[] \textcolor{dark_red}{\textsf{R}: mode.insead()}
\item[] But we have to write the mode function first: 
\begin{verbatim}
mode.insead <- function(x) {
  ux <- unique(x)
  ux[which.max(tabulate(match(x, ux)))]
}
\end{verbatim}
\item Feel free to copy/paste!! 
\end{itemize}
\end{scriptsize}
\end{frame}

\begin{frame}[fragile]
\frametitle{Measures of location}
\begin{scriptsize}
\begin{itemize}
\item Go to: \href{https://rdrr.io/snippets/}{\textcolor{insead_light}{https://rdrr.io/snippets/}} 
\item Consider: \begin{verbatim} mydata <- c(1, 2, 3, 3, 4, 5, 6) \end{verbatim}
\item[] [the simplest type of data structure in \textsf{R}]
\item Find the mean, median, and mode of mydata. Hint: for mode, you will have to write in the function first. 
\item Here it is again: 
\begin{verbatim}
mode.insead <- function(x) {
  ux <- unique(x)
  ux[which.max(tabulate(match(x, ux)))]
}
\end{verbatim}
\item Solution:
\begin{verbatim}  
mode.insead <- function(x) {
       ux <- unique(x)
       ux[which.max(tabulate(match(x, ux)))]
}

mydata <- c(1, 2, 3, 3, 4, 5, 6)

print(c(mean(mydata), median(mydata), mode.insead(mydata)))
[1] 3.428571 3.000000 3.000000
\end{verbatim}
\item Note: We instruct \textcolor{dark_red}{\textsf{R}} to print a vector just to be concise. You \emph{can} just plug in mean, mode, and median and run it. 
\end{itemize}
\end{scriptsize}
\end{frame}

\begin{frame}[fragile]
\frametitle{Example}
\begin{scriptsize}
Given the following data: 

$$ 10, 3, 2, 15, 1, 3, 4, 5, 8, 2, 12, 20, 3, 5, 10 $$

compute the mean, median and mode by hand, then verify in \textcolor{dark_red}{\textsf{R}}. 

\pause 
\vspace{0.5cm}
Solution: 
\begin{verbatim}
mode.insead <- function(x) {
       ux <- unique(x)
       ux[which.max(tabulate(match(x, ux)))]
}

mydata <- c(10,3,2,15,1,3,4,5,8,2,12,20,3,5,10)

print(c(round(mean(mydata),2), round(median(mydata),2), 
round(mode.insead(mydata),2)))
[1] 6.87 5.00 3.00
\end{verbatim} 
We have also instructed \textcolor{dark_red}{\textsf{R}} to round the output to two decimal points.
\end{scriptsize}
\end{frame}

\begin{frame}[fragile]
\frametitle{Measures of dispersion}
\begin{scriptsize}
How spread out is my data?
\begin{itemize}
\item Maximum, minimum, range
\item[] \textcolor{dark_red}{ \textsf{R}: max(), min(), max() - min() }
\item Variance 
\item[] \textcolor{dark_red}{ \textsf{R}: var.insead() }
\begin{verbatim}
var.insead = function(x){var(x)*(length(x)-1)/length(x)}
\end{verbatim}
\item Standard deviation 
\item[] \textcolor{dark_red}{\textsf{R}: sqrt(var.insead()) }
\end{itemize}
\end{scriptsize}
\end{frame}



\begin{frame}
\frametitle{Dispersion on an example: oil prices}
\begin{center}
\hspace*{-0.55cm} \includegraphics[scale=0.5]{disp_1.png}
\end{center}
\end{frame}

\begin{frame}
\frametitle{Dispersion on an example: oil prices}
\begin{center}
\hspace*{-0.55cm} \includegraphics[scale=0.5]{disp_2.png}
\end{center}
\end{frame}

\begin{frame}
\frametitle{Dispersion on an example: oil prices}
\begin{center}
\hspace*{-0.55cm} \includegraphics[scale=0.5]{disp_3.png}
\end{center}
\end{frame}

\begin{frame}
\frametitle{Variance and standard deviation}
\begin{itemize}
	\itemsep\setlength{1em}
\item We will often use $\mu$ to denote the mean, and $\sigma$ to denote standard deviation. 
\item Variance (when all outcomes are equally likely):
\item[] $$ \textrm{Var} = \frac{1}{n} \sum_{i=1}^n (x - \mu)^2 $$
\item Standard deviation: $\sigma = \sqrt{\textrm{Var}} $
\item[] [So: $\textrm{Var} = \sigma^2$]
\item Rougly speaking, 95\% of the data will be contained in the interval spanning $\pm 2$ standard deviations from the mean. 
\end{itemize}
\end{frame}

\begin{frame}[fragile]
\frametitle{Exercise}
Consider the following data, drawn from a uniform distribution in \textcolor{dark_red}{\textsf{R}} using the command \textcolor{blue}{round(runif(10)*10, 0)}:

$$ 8,  1,  2,  4,  8,  6,  2,  6,  9, 2 $$

Find mean, variance, and standard deviation. Note:

\begin{verbatim}
var.insead = function(x)
    {var(x)*(length(x)-1)/length(x)}
\end{verbatim}

\end{frame}

\begin{frame}[fragile]
\frametitle{Solution}
\begin{verbatim}
var.insead = function(x)
    {var(x)*(length(x)-1)/length(x)}

mydata <- c(8,1,2,4,8,6,2,6,9,2)

mean(mydata) 

var.insead(mydata) 

sqrt(var.insead(mydata))

[1] 4.8
[1] 7.96
[1] 2.821347
\end{verbatim}
\end{frame}

%endnotes
\begin{frame}
\frametitle{Today}
\begin{scriptsize}
\begin{tabular}{ll}
\textcolor{fade}{Basics} & \begin{tabular}[c]{@{}l@{}}\textcolor{fade}{Functions}\\ \textcolor{fade}{Linear}\\ \textcolor{fade}{Inverse}\\ \textcolor{fade}{Two equations}\\ \textcolor{fade}{Quadratic}\end{tabular} \vspace{0.25em} \\ 

\textcolor{fade}{Exponents}   & \begin{tabular}[c]{@{}l@{} }\arrayrulecolor{fade}\hline \\ \textcolor{fade}{ Exponents}\\ \textcolor{fade}{ Application: interest rates}\\ \textcolor{fade}{Exponential functions}\\ \textcolor{fade}{Logarithmic functions}\end{tabular} \vspace{0.25em} \\

\textcolor{fade}{Logarithms} & \begin{tabular}[c]{@{}l@{}} \\ \arrayrulecolor{fade}\hline \\ \textcolor{fade}{Logarithmic functions}\\ \textcolor{fade}{Logarithmic and exponential equations} \\ \textcolor{fade}{Case: pricing}\\ \textcolor{fade}{Derivatives} \end{tabular}   \vspace{0.25em}  \\

\textcolor{black}{Derivatives}  & \begin{tabular}[c]{@{}l@{}} \arrayrulecolor{fade}\hline \\ \textcolor{black}{Optimal decisions}\\ \textcolor{black}{Case: production}\\\textcolor{black}{ Statistics}\end{tabular}        \vspace{0.25em} \\

\textcolor{fade}{Uncertainty}    & \begin{tabular}[c]{@{}l@{}} \arrayrulecolor{fade}\hline \\\textcolor{fade}{ Probability \& statistics}\\ \textcolor{fade}{Normal distribution }\end{tabular}     \vspace{0.25em} \\       
\end{tabular}
\end{scriptsize}
\end{frame}


    {\setbeamercolor{background canvas}{bg=insead_dark}
    \begin{frame}[plain]
    \begin{tikzpicture}[remember picture,overlay]
    \node[at=(current page.center)] {
    \includegraphics[keepaspectratio,
    width=\paperwidth,
    height=\paperheight]{/Users/borisbabic/_INSEAD_teaching/_business_foundations/_syllabus/end_slide.png}};
    \end{tikzpicture}
    \end{frame}}
\end{document}

\begin{frame}
\frametitle{}

\begin{itemize}
    \itemsep\setlength{1em}
    \item 
    \item 
    \item 
    \item 
\end{itemize}
\end{frame}

\end{frame}