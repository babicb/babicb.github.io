\documentclass[xcolor=dvipsnames, 9pt]{beamer} % dvipsnames gives more built-in colors

% theme and color settings
\usetheme{Madrid}
\useoutertheme[hideothersubsections, left, height=0pt]{sidebar}
\useinnertheme{circles}

\definecolor{insead_dark}{RGB}{0,110,91} 
\definecolor{insead_light}{RGB}{160,206,103}
\definecolor{fade}{HTML}{C1C7D0}
\definecolor{light_red}{HTML}{DCBCBC}
\definecolor{mid_red}{HTML}{B97C7C}
\definecolor{dark_red}{HTML}{8F2727}

\usecolortheme[named=insead_dark]{structure}
\usefonttheme[onlymath]{serif}

% some packages 
\usepackage{xcolor}
\usepackage{graphicx} 
\usepackage{booktabs} 
\usepackage{multicol}
\usepackage{tikz}
\usepackage{setspace}
\usepackage{tcolorbox}
\usepackage{colortbl}

%frontmatter
\title[Session 2 (Exponents)]{\normalsize{MBA Business Foundations, \\Quantitative Methods: \\Session Two}}
\author[Boris Babic, INSEAD]{Boris Babic, \\Assistant Professor of Decision Sciences}
\institute[]{}
\date{}

%\beamerdefaultoverlayspecification{<+->}

\begin{document}

\setbeamertemplate{sidebar left}{}
\begin{frame}
\titlepage
\vspace{-5em}
\begin{center}
\includegraphics[scale=0.4]{/Users/borisbabic/_INSEAD_teaching/_business_foundations/_syllabus/insead_logo.png}
\end{center}
\end{frame}


\begin{frame}
\frametitle{Today}
\begin{small}
\begin{tabular}{ll}
\textcolor{fade}{Basics} & \begin{tabular}[c]{@{}l@{}}\textcolor{fade}{Functions}\\ \textcolor{fade}{Linear}\\ \textcolor{fade}{Inverse}\\ \textcolor{fade}{Two equations}\\ \textcolor{fade}{Quadratic}\end{tabular} \vspace{0.25em} \\ 

\textcolor{insead_dark}{Exponents}   & \begin{tabular}[c]{@{}l@{} }\arrayrulecolor{fade}\hline \\ \textcolor{black}{ Exponents}\\ \textcolor{black}{ Application: interest rates}\\ \textcolor{black}{Exponential functions}\\ \textcolor{black}{Logarithmic functions}\end{tabular} \vspace{0.25em} \\

\textcolor{fade}{Logarithms} & \begin{tabular}[c]{@{}l@{}} \arrayrulecolor{fade}\hline \\ \textcolor{fade}{Logarithmic functions}\\ \textcolor{fade}{Logarithmic and exponential equations} \\ \textcolor{fade}{Case: pricing}\\ \textcolor{fade}{Derivatives} \end{tabular}   \vspace{0.25em}  \\

\textcolor{fade}{Derivatives}  & \begin{tabular}[c]{@{}l@{}} \arrayrulecolor{fade}\hline \\ \textcolor{fade}{Optimal decisions}\\ \textcolor{fade}{Case: production}\\\textcolor{fade}{ Statistics}\end{tabular}        \vspace{0.25em} \\

\textcolor{fade}{Uncertainty}    & \begin{tabular}[c]{@{}l@{}} \arrayrulecolor{fade}\hline \\\textcolor{fade}{ Probability \& statistics}\\ \textcolor{fade}{Normal distribution }\end{tabular}     \vspace{0.25em} \\       
\end{tabular}
\end{small}
\end{frame}

\setbeamertemplate{sidebar left}[sidebar theme]
\section{Exponents}

\begin{frame}
\frametitle{Exponents}
\begin{itemize}
\item Essential for analysis of interest rates and growth. 
\item Denotes repeated multiplication of the same quantity 
\end{itemize}
\begin{center}
\includegraphics[scale=0.5]{expo_def.png}
\end{center}
\end{frame}

\begin{frame}
\frametitle{Exponents}
\begin{itemize}
\item[] Examples:
\item $3^4$
\item $1.5^2$
\item $\big( \frac{1}{2} \big)^2 $
\item $(a + 2)^2 = $
\item[] Solutions:
\item $3^4 = 81$
\item $1.5^2 = 2.25$
\item $\big( \frac{1}{2} \big)^2 = \big( \frac{1}{4} \big)^2 $
\item $(a + 2)^2 = a^2 + 4a + 4 $ (what kind of function?)
\end{itemize}
\end{frame}

\begin{frame}
\frametitle{Some examples of exponential phenomena}
\begin{center}
\includegraphics[scale=0.25]{expo_life.png}
\end{center}
\end{frame}

\begin{frame}
\frametitle{Rules for exponents}

\begin{table}[]
\begin{tabular}{ll}
Products & $b^c \cdot b^d = b^{c+d} $ \\
\\ Powers & $(b^c)^d = (b^d)^c = b^{c \cdot d}$ \\
\\ Negative exponents & $b^{-c} = \frac{1}{b^c}$ \\
\\ Quotients & $\frac{b^c}{b^d} = b^{c-d}$ \\
\\ Zero power & $b^0 = 1$ \\
\\ Roots & $\sqrt[n]{b} = b^{1/n}$
\end{tabular}
\end{table}
\end{frame}

\begin{frame}
\frametitle{The formulas in practice}
Simplify the following: 

\begin{itemize}
\setlength\itemsep{1em}
\item $(3^4)^2 =$ 
\item $\frac{6^2}{6^5} =$ 
\item $3^0 =$ 
\item $27^{2/3} =$  
\item $\bigg( \frac{x}{y} \bigg)^3 \cdot \bigg( \frac{x}{z} \bigg)^{-2}$
\item $\frac{x^3y^2}{x^5y^{-2}}$
\item $\frac{24x^5 y^3 z^7}{6x^3 y^2 z^4}$
\end{itemize}
\end{frame}

\begin{frame}
\frametitle{Solutions}
\begin{itemize}
    \itemsep\setlength{1em}
\item $(3^4)^2 = 3^8$ 
\item $\frac{6^2}{6^5} = 6^{2-5} = 6^{-3} = \frac{1}{6^3}$ 
\item $3^0 = 1$ 
\item $27^{2/3} = \sqrt[3]{27^2}$  
\item $\bigg( \frac{x}{y} \bigg)^3 \cdot \bigg( \frac{x}{z} \bigg)^{-2} = \bigg( \frac{x}{y} \bigg)^3 \cdot \bigg( \frac{z}{x} \bigg)^{2} = \frac{x^3z^2}{y^3x^2} = \frac{xz^2}{y^3}$
\item $\frac{x^3y^2}{x^5y^{-2}} = \frac{x^3y^2y^2}{x^5} = x^{-2}y^4 = \frac{y^4}{x^2}$
\item $\frac{24x^5 y^3 z^7}{6x^3 y^2 z^4} = 4x^2yz^3$
\end{itemize}
\end{frame}

%\item $\frac{5x^6y^3}{4w^4z^2} \div \frac{8wz^3}{15x^2y} =$ 
%\item $\frac{6xy^5}{5w^4z^4} \div \frac{2x^3y}{10w^3z^2} =$

\section{Application: interest rates}

\begin{frame}
\frametitle{Simple interest, part 1}
\begin{itemize}
\item $I$: interest income in one period 
\item $P$: capital to invest
\item $r$: interest rate per period 
\item $n$: number of periods invested
\item $P = \$1,000$, $r = 4\%$, what is $I$ after one year?
\item Solution: $\$40$
\item if $n = 3$, what is $I$?
\item Solution $\$120$
\item In general, $I = P \cdot r \cdot n$
\end{itemize}
\end{frame}


\begin{frame}
\frametitle{Simple interest, part 2}

Ex: Treasury notes. 

\begin{center}
\hspace*{-0.4cm} \includegraphics[scale=0.45]{interest2.png}
\end{center}
%\pause
\vspace{-1em} 
NOTE: If we assume annual interest (the part in yellow is just about how you are receiving your earnings):
$5000 + 10 \cdot 0.08 \cdot 5000 - 4500 = 5000 + 10 \cdot 400 - 4500 = 4500$. 

\end{frame}

\begin{frame}
\frametitle{Compound interest, part 1}
\begin{itemize}
\item (Capital increases by interest every period)
\item You invest $\$1$ at an annual interest rate $r=4\%$. 
\item After year $1$: $1 + 1(0.04) = 1.04$
\item After year $2$: $1.04 + 1.04(0.04) = 1.04(1 + 0.04) = 1.04^2$
\item After year $3$: $1.04^2 + 1.04^2(0.04) = 1.04^2(1 + 0.04) = 1.04^3$
\item After year $t$: $1.04^t$
\item[] Some notation: 
\item $P=$ present amount
\item $A=$ final amount
\item $r =$ interest rate
\item $t=$ number of years money is invested. 
\item General formula: if compounding annually, \textcolor{dark_red}{$A = P(1 + r)^t$}
\end{itemize}

\end{frame}

\begin{frame}
\frametitle{Compound interest, part 2}
Compounding could be done: 

\begin{itemize}
\item Yearly $\rightarrow$ rate/period $=r$
\item Semi-annually $\rightarrow$ rate/period $=r/2$
\item Quarterly $\rightarrow$ rate/period $=r/4$
\item Monthly $\rightarrow$ rate/period $=r/12$
\item Some more notation: $n$ number of periods per year
A more general formula: \textcolor{dark_red}{$A = P\big(1 + \frac{r}{n}\big)^{tn}$}
\end{itemize}
\end{frame}

\begin{frame}
\frametitle{Compound interest, part 3}

A general \textcolor{dark_red}{inverse} formula: if we know the final amount $A$, the interest rate $r$, the time money is invested $t$ and compounding periods per year, $n$, we can calculate the principal $P$. 

\textcolor{dark_red}{$$ P = A \bigg(1 + \frac{r}{n}\bigg)^{-tn} $$}

\end{frame}

\begin{frame}
\frametitle{Examples on interest rates, part 1}
\begin{itemize}
\item $P = \$1000$, $r = 4\%$, $t = 3$ years
\item Compare the final amount $A$ for,
\item Simple interest
\item Compounded annually
\item Compounded semi-annually 
\item Compounded quarterly 
\item Compounded monthly 
\item [] Answers: 
\item $1000 \cdot 0.04 \cdot 3 = 120 \rightarrow 1000 + 120 = 1120$
\item $A = P(1 + r)^t \rightarrow 1000(1 +0.04)^3 = 1124.87$
\item $A = P(1 + r/n)^{tn} = 1000(1+ 0.04/2)^{3 \cdot 2} = 1126.12$
\item $A = P(1 + r/n)^{tn} = 1000(1+ 0.04/4)^{3 \cdot 4} = 1126.83$
\item $A = P(1 + r/n)^{tn} = 1000(1+ 0.04/12)^{3 \cdot 12} = 1127.27$
\end{itemize}
\end{frame}

\begin{frame}
\frametitle{Examples on interest rates, part 2}
\begin{itemize}
\item Problem 1: I borrowed $\$2,000$ for $5$ years at $r = 8\%$, compounded quarterly. How much do I have to pay back at the end of the term? 
\item Problem 2: I put my money in a savings account at $r = 6\%$ which is compounded semi-annually and received $\$530.45$ at the end of the year. How much did I put in at the beginning? 
\item Solution 1: $2000(1 + 0.08/4)^{5 \cdot 4} = 2971.89$
\item Solution 2: $P = A(1 + r/n)^{-tn} = 530.45(1+0.06/2)^{-2} = 500$
\end{itemize}
\end{frame}

\section{Exponential functions}

\begin{frame}
\frametitle{Exponential functions}

$f(x) = b^x$ where $b > 0$, $b$ is the base, $x$ is the exponent. \\ (Check: is $x^2$ exponential? Why or why not?)

\begin{center}
\includegraphics[scale=0.4]{expo_function1.png}
\includegraphics[scale=0.42]{expo_function2.png}
\end{center}
\end{frame}

\begin{frame}
\frametitle{Exponential functions}
\begin{itemize}
\item All the curves pass through the point $(x, y) = (0, 1)$. 
\item The exponential functions are always above the $f(x) = 0$ horizontal line. In fact, that line is an asymptote. 
\item $f(x) \rightarrow 0$ as $x \rightarrow - \infty$ (when $b > 1$), and $f(x) \rightarrow 0$ as $x \rightarrow \infty$ when $0 < b < 1$. 
\item Can we have $b < 0$? Consider $f(x) = -4^x$. What is $f(2)$?, $f(-3)$?, $f(1/2)$? 
\item[] (hint: $\sqrt{-4} = 2i$). 
\end{itemize}
\end{frame}

\begin{frame}
\frametitle{Exponential functions}
\begin{itemize}
\item If $b > 1$, the curve becomes steeper as $b$ increases
\item if $0 < b < 1$, it is the other way around, the curve becomes steeper as the base gets closer to $0$. 
\end{itemize}

\begin{center}
\includegraphics[scale=0.6]{expo_curves.png}
\end{center}

\end{frame}

\begin{frame}
\frametitle{Practice}
Match each equation with the graph of $f, g, h, k$: 
%\pause
\begin{itemize}
\item[](A) $f(x) = 2^x$
\item[](B) $f(x) = (0.2)^x$
\item[](C) $f(x) = 4^x$
\item[](D) $f(x) = (1/3)^x$
\end{itemize}

\begin{center}
\includegraphics[scale=0.3]{practice_curves.png}
\end{center}
\end{frame}

\begin{frame}
\frametitle{A generalized example}
\begin{itemize}
\item Exponential functions can be generalized to $f(x) = ab^x$ where $a$ is now a scaling constant of the function. 
\item Ex: Compound interest when compounded annually. 
\item[] Recall it is given by $A = P(1+r)^t$. Here $a = P$, $b = (1+r)$, and $x=t$.
\item Ex: Compound interest when compounded $n$ times per year. 
\item[] Given by $A = P\big( 1 + \frac{r}{n} \big) ^{tn}$
\item Why is this an exponential function of the form $f(x) = ab^x$?
\item[] $A = P \bigg[ \big( 1 + \frac{r}{n} \big) ^{n} \bigg] ^t$
\item[] Here $a = P, b = \big( 1 + \frac{r}{n} \big)^n, x = t$. 
\end{itemize}
\end{frame}

\begin{frame}
\frametitle{A very special case}
\begin{itemize}
\item $f(x) = e^x$, where $e \approx 2.71828$, named after Leonhard Euler. 
\end{itemize}
\begin{center}
\includegraphics[scale=0.5]{eulers_curve.pdf}
\end{center}
\vspace{-2em}
\begin{itemize}
\item Examples: 
\item Finance: continuous compounding: $A = P \cdot e^{rt}$
\item Economics: growth rate: $e^{0.03t}$
\item Probability: exponential families (includes normal distribution!)
\end{itemize}
\end{frame}

\begin{frame}
\frametitle{Example: growth/decay}

\begin{center}
\includegraphics[scale=0.5]{growth_decay.png}
\end{center}
\begin{itemize}
    \item $A = A_0e^{kt} $, where $A=$ ending value, $A_0=$ initial value, $t$ is elapsed time, and $k$ is the growth/decay rate. 
    \item $k > 0$, the amount is increasing (growing); $k < 0$, the amount is decreasing (decaying).
    \item Ex: bacteria grow continuously -- i.e., they do not ``wait'' and then all at once reproduce in the next period.
\end{itemize}
\end{frame}

\begin{frame}
\frametitle{Example: Carbon decay}
\begin{itemize}
\item Problem: A certain artifact originally had 12 grams of carbon-14 present.  Suppose the decay model $A = 12e^{-0.000121t}$ correctly describes the amount of carbon-14 present after $t$ years.  How many grams of carbon-14 will be present in this artifact after $10,000$ years? 
\item[] \begin{align*}
A 
&= 12e^{-0.000121t} \\
&= 12e^{-0.000121 \cdot 10000} \\
&= 3.58
\end{align*}
\end{itemize}
\end{frame}

\begin{frame}
\frametitle{Example: Bacteria growth}
\begin{itemize}
\item Problem: A strain of bacteria growing on your desktop doubles every 5 minutes. Assuming that you start with only one bacterium, how many bacteria could be present after 1.5 hours? Hint: $ \log(e^x) = x$.
\item[] \begin{align*}
A
&= A_0 e^{kt} \\
&\rightarrow 2 = 1\cdot e^{k \cdot 5} \\
&\rightarrow 2 = e^{5k} \\
&\rightarrow \log 2 = \log e^{5k} \\
&\rightarrow \log 2 = 5k \\
&\rightarrow \frac{\log 2}{5} = k\\
&\rightarrow k = 0.139 \\
&\rightarrow A = A_0e^{0.139 \cdot 90} \\
&\rightarrow A = 1 \cdot e^{0.139 \cdot 90} = 271,034!
\end{align*}
\end{itemize}
\end{frame}

\begin{frame}
\frametitle{Example: population growth}


The expression $A=30\exp ( 0.019t )$ \textcolor{red}{(note: $\exp(x) = e^x)$} describes the population of a city, in thousands, $t$ years after 2015.  Use this expression to solve the following:

\begin{itemize}
\item What was the population of the city in 2015?
\item By what \% is the population of the city increasing each year?
\item What will the population be in 2026? 
\item When will the city’s population be 60 thousand?
\item[] Solutions
\item Set $t = 0$, convert to thousands $=30,000 = A_0$.
\item It is increasing $k \cdot 100 = 1.9\%$ each year.
\item $30\exp\{0.019*11\} \approx 37 \rightarrow 37 \cdot 1000 = 37000$.
\item $60 = 30e^{0.019t} \rightarrow t = 36 \rightarrow 2015 + 36 = 2051$.

\end{itemize}

\end{frame}

\section{Logarithmic functions}

\begin{frame}
\frametitle{Logarithmic functions}
\begin{itemize}
\item A logarithmic function is the \textcolor{dark_red}{inverse} of an exponential function 
\item If $b^x = c$ then $\log_b(c) = x$ 
\item Natural log: if $e^x = c$ then $\ln (c) = x$ where $\ln x = \log_e x$
\end{itemize}
\begin{center}
\includegraphics[scale=0.5]{log_exp.png}
\end{center}
\end{frame}

\begin{frame}
\frametitle{Logarithmic functions}
\begin{itemize}
\item There are no logs of zero or negative numbers $( x > 0)$ (Why?).
\item If $\log_b(-k) = c$ then $b^c = -k$. 
\item Logs of numbers less than one are negative. 
\item All curves pass through the point $(x, y) = (1, 0)$.
\item When $x$ tends to $0$ in positive value, $f(x)$ is higher and higher in negative value.
\item The vertical line at $x = 0$ is an asymptote: a straight line which the graph approaches but never touches. 
\end{itemize}
\end{frame}

\begin{frame}
\frametitle{Operations on logs}
\begin{multicols}{2}
\begin{itemize}
\setlength\itemsep{1em}
\item $\log_b(b^x) = x$
\item $b^{\log_b(x)}=x$
\item $\log_b(c \cdot d) = \log_b c + \log_b d$
\item $\log_b \frac{c}{d} = \log_b c - \log_b d$
\item $\log_b(c^d) = d \cdot \log_b c$
\item $\log_b(b) = 1$
\item $\log_b 1 = 0$
\end{itemize}
\end{multicols}
\begin{center}
\hspace*{-1cm} \includegraphics[scale=0.5]{natural_log.pdf}
\end{center}
\end{frame}

\begin{frame}
\frametitle{Practice examples}

\begin{itemize}
	\setlength\itemsep{1em}
\item[] Practice:
\item Ex 1: $\log_5(4x+11) = 2$
\item Ex 2: $\log_2(x+5) - \log_2(2x-1) =5$
\item Ex 3: $\log_8(x) + \log_8(x+6) = \log_8(5x+12)$ 
\item[] Hint: get into quadratic form, find positive root
\item Ex 4: $\log_6(x) + \log_6(x - 9) = 2$
\item Ex 5: $\ln (10) - \ln (7-x) = \ln (x)$
\end{itemize}
\end{frame}

\begin{frame}
\frametitle{Ex 1 Solution}
\begin{align*}
\log_5(4x+11) &= 2 \\
4x + 11 &= 5^2 \\
x &= 7/2 
\end{align*}
\end{frame}

\begin{frame}
\frametitle{Ex 2 Solution}
\begin{align*}
\log_2(x+5) - \log_2(2x-1) &= 5 \\
\log_2 \bigg( \frac{x+5}{2x-1} \bigg) &= 5 \\
\bigg( \frac{x+5}{2x-1} \bigg) &= 2^5 = 32 \\
x + 5 &= 32(2x-1) \\
x &= 37/63
\end{align*}
\end{frame}

\begin{frame}
\frametitle{Ex 3 Solution}
\begin{align*}
\log_8(x) + \log_8(x+6) &= \log_8(5x+12) \\
\log_8(x(x+6)) &= \log_8 (5x + 12) \\
x(x+6) &= 5x+12 \\
x^2 + 6x &= 5x + 12 \\
x^2 + x - 12 &= 0 \\
(x - 3)(x +4) &= 0 \rightarrow x = 3
\end{align*}
\end{frame}

\begin{frame}
\frametitle{Ex 4 Solution}
\begin{align*}
\log_6(x) + \log_6(x - 9) &= 2 \\
x(x-9) &= 36 \\
(x +3)(x - 12) &= 0 \\
x &= 12 
\end{align*}
\end{frame}

\begin{frame}
\frametitle{Ex 5 Solution}
\begin{scriptsize}
\begin{align*}
\ln(10) - \ln(7-x) &= \ln(x) \\
\ln \bigg( \frac{10}{7-x} \bigg) &= \ln x \\
\bigg( \frac{10}{7-x} \bigg) &= x \\
10 &= x(7-x) \\
10 &= 7x - x^2 \\
x^2 - 7x + 10 &= 0 \\
(x - 5)(x-2) &= 0\\ 
\rightarrow x = 2, x = 5
\end{align*}
\end{scriptsize}
\begin{center}
\includegraphics[scale=0.42]{ex6.png}
\end{center}
\end{frame}

%endnotes
\begin{frame}
\frametitle{Today}
\begin{small}
\begin{tabular}{ll}
\textcolor{fade}{Basics} & \begin{tabular}[c]{@{}l@{}}\textcolor{fade}{Functions}\\ \textcolor{fade}{Linear}\\ \textcolor{fade}{Inverse}\\ \textcolor{fade}{Two equations}\\ \textcolor{fade}{Quadratic}\end{tabular} \vspace{0.25em} \\ 

\textcolor{insead_dark}{Exponents}   & \begin{tabular}[c]{@{}l@{} }\arrayrulecolor{fade}\hline \\ \textcolor{black}{ Exponents}\\ \textcolor{black}{ Application: interest rates}\\ \textcolor{black}{Exponential functions}\\ \textcolor{black}{Logarithmic functions}\end{tabular} \vspace{0.25em} \\

\textcolor{fade}{Logarithms} & \begin{tabular}[c]{@{}l@{}} \arrayrulecolor{fade}\hline \\ \textcolor{fade}{Logarithmic functions}\\ \textcolor{fade}{Logarithmic and exponential equations} \\ \textcolor{fade}{Case: pricing}\\ \textcolor{fade}{Derivatives} \end{tabular}   \vspace{0.25em}  \\

\textcolor{fade}{Derivatives}  & \begin{tabular}[c]{@{}l@{}} \arrayrulecolor{fade}\hline \\ \textcolor{fade}{Optimal decisions}\\ \textcolor{fade}{Case: production}\\\textcolor{fade}{ Statistics}\end{tabular}        \vspace{0.25em} \\

\textcolor{fade}{Uncertainty}    & \begin{tabular}[c]{@{}l@{}} \arrayrulecolor{fade}\hline \\\textcolor{fade}{ Probability \& statistics}\\ \textcolor{fade}{Normal distribution }\end{tabular}     \vspace{0.25em} \\       
\end{tabular}
\end{small}
\end{frame}


	{\setbeamercolor{background canvas}{bg=insead_dark}
    \begin{frame}[plain]
    \begin{tikzpicture}[remember picture,overlay]
    \node[at=(current page.center)] {
    \includegraphics[keepaspectratio,
    width=\paperwidth,
    height=\paperheight]{/Users/borisbabic/_INSEAD_teaching/_business_foundations/_syllabus/end_slide.png}};
    \end{tikzpicture}
    \end{frame}}
\end{document}